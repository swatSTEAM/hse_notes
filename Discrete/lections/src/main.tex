
\documentclass[a4paper,12pt]{article} % добавить leqno в [] для нумерации слева
\usepackage[top=2 cm, bottom=2cm, left=2cm, right=2cm]{geometry}
%%% Работа с русским языком
\usepackage{cmap}					% поиск в PDF
\usepackage{mathtext} 				% русские буквы в формулах
\usepackage[T2A]{fontenc}			% кодировка
\usepackage[utf8]{inputenc}			% кодировка исходного текста
\usepackage[english,russian]{babel}	% локализация и переносы

%%% Дополнительная работа с математикой
\usepackage{amsmath,amsfonts,amssymb,amsthm,mathtools} % AMS
\usepackage{icomma} % "Умная" запятая: $0,2$ --- число, $0, 2$ --- перечисление

%% Номера формул
%\mathtoolsset{showonlyrefs=true} % Показывать номера только у тех формул, на которые есть \eqref{} в тексте.

%% Шрифты
\usepackage{euscript}	 % Шрифт Евклид
\usepackage{mathrsfs} % Красивый матшрифт
\usepackage{fancyhdr}
\pagestyle{fancy}
\lhead{Вдовкин В.А., Юткин Д.И.}
\rhead{БИВ-141}

%% Свои команды\textbf{}
\DeclareMathOperator{\sgn}{\mathop{sgn}}

%% Перенос знаков в формулах (по Львовскому)
\newcommand*{\hm}[1]{#1\nobreak\discretionary{}
{\hbox{$\mathsurround=0pt #1$}}{}}
\usepackage{indentfirst}
% \setlength{\parindent}{1.25cm}



\usepackage{tikz}
\usepackage{tkz-graph}
\usetikzlibrary{decorations.pathreplacing}
\usepackage{caption}
\usepackage{float}
\usepackage{wrapfig}
\usepackage{multicol}




\usepackage{mathrsfs}
\usetikzlibrary{arrows}
\usetikzlibrary{shapes,arrows}
% \usepackage{pgfplots}
% \pgfplotsset{compat=1.9}
\tikzstyle{decision} = [diamond, draw, fill=blue!20, 
    text width=4.5em, text badly centered, node distance=3cm, inner sep=0pt]
\tikzstyle{block} = [rectangle, draw, fill=blue!20, 
    text width=7em, text centered, rounded corners, minimum height=2em]
\tikzstyle{block2} = [rectangle, draw, fill=yellow!20, 
    text width=7em, text centered, rounded corners, minimum height=2em]
\tikzstyle{line} = [draw, -latex']
\tikzstyle{cloud} = [draw, ellipse,fill=red!20, node distance=4cm,
    minimum height=2em]

\usepackage{hyperref}

\hypersetup{ % Гиперссылки
    unicode=true, % русские буквы в раздела PDF
    pdftitle={Заголовок}, % Заголовок
    pdfauthor={Автор}, % Автор
    pdfsubject={Тема}, % Тема
    pdfcreator={Создатель}, % Создатель
    pdfproducer={Производитель}, % Производитель
    pdfkeywords={keyword1} {key2} {key3}, % Ключевые слова
    colorlinks,
    linkcolor={black},
    citecolor={black},
    urlcolor={black}
}

\begin{document}
\tableofcontents
\clearpage

\setcounter{section}{6}
\section{Лекция 7}
    \subsection{Граф}
    Графом $G = (V, E)$ называется пара конечных множеств $V$ и $E$, где:
    \begin{align*}
        V &= \{v_1, v_2, ..., v_n\} \text{ --- множество вершин, } |V| = n,\\
        E &= \{e_1, e_2, ..., e_m\} \text{ --- множество рёбер, } |E| = m.
    \end{align*}
    
    Если $e \in E \Rightarrow e =\{ v_i, v_j \}$ - неупорядоченная пара.
    
    \subsubsection{Пример задания графа диаграмой}
    \begin{center}
        $G = (V, E)$, где $V = \{v_1, v_2, v_3\}$; $E = \big\{ \{v_1, v_2\}, \{v_1, v_3\}, \{v_2, v_2\}, \{v_2, v_2\}, \{v_1, v_3\} \big\}$.
        
        \begin{tikzpicture}
            \SetGraphUnit{1}
            \Vertex[Math] {v_1}
            \Vertex[Math, x=3,y=0]{v_2}
            \Vertex[Math, x=1.5,y=-1.5]{v_3}
            \Edge[style={-}	](v_1)(v_2)
            \Edge[style={-}](v_1)(v_3)
            \Loop[dist=0.8cm,dir=SO,style={-}](v_2)
            \Loop[dist=1.5cm,dir=EA,style={-}](v_2)
            \tikzset{EdgeStyle/.style = {-,bend left}}
            \Edge(v_1)(v_3)
        \end{tikzpicture}

    \end{center}
    
    \subsection{Инцидентность}
    Инцидентность --- понятие, используемое только в отношении ребра и вершины: если $v_1$, $v_2$ --- вершины, а $e=\{v_1,v_2\}$ --- соединяющее их ребро,
    тогда вершина $v_1$ и ребро e инцидентны, вершина $v_2$ и ребро $e$ тоже инцидентны. Две вершины (или два ребра) инцидентными быть не могут.
    
    \subsection{Петля}
    Петля в графе --- ребро, инцидентное одной и той же вершине (ребро, у которого вершины одинаковые).
    
    \subsection{Кратные рёбра}
    Кратные рёбра --- несколько рёбер, инцидентных одной и той же паре вершин, т.е. если $e = \{ v_i, v_j \}$ --- встречается несколько раз
    (если один раз, то ребро называется простым).
    
    \subsection{Простой граф}
    Граф, в котором нет петель и кратных рёбер называется простым.

    \subsection{Степень вершины}
    \begin{multicols}{2}
          Дан граф $G = (V, E)$, $v \in V$. Степень вершины ($\deg(v)$) - количество рёбер, выходящих из вершины.
          \begin{figure}[H]
              \vspace*{-20pt}
              \centering
              \begin{tikzpicture}
              \SetGraphUnit{1}
              \Vertex[Math] {v_0}
              \Vertex[Math, x=3,y=1]{v_1}
              \Vertex[Math, x=3,y=0]{v_2}
              \Vertex[Math, x=3,y=-1]{v_3}
              \Edge[style={-}	](v_0)(v_1)
              \Edge[style={-}](v_0)(v_2)
              \Edge[style={-}](v_0)(v_3)
              \Loop[dist=1.5cm,dir=WE,style={-}](v_0)
              \tikzset{EdgeStyle/.style = {-,bend right}}
              \Edge(v_0)(v_3)
              \end{tikzpicture}
              \caption*{$\deg(v_0) = 6$ (петля учитывается дважды)}
            \end{figure}
    \end{multicols}
    
    \subsection{Изолированная вершина}
    Вершина, степень которой равна $0$ ($\deg(v) = 0$) называется изолированной.
    
    \subsection{Висячая вершина}
    Вершина, степень которой равна $1$ ($\deg(v) = 1$) называется висячей.
    
    \subsection{Лемма о рукопожатиях} \label{handsshake}
    Рассмотрим граф $G = (V, E)$; $|V| = n$, $|E| = m$.
    \[ \displaystyle\sum_{i=1}^{n} \deg(v_i) = 2m = 2|E| \]
    то есть сумма степеней вершин любого графа равна удвоенному числу его рёбер. Данное утверждение (и сама формула) известны как лемма
    о рукопожатиях. Название происходит от известной математической задачи: необходимо доказать,
    что в любой группе число людей, пожавших руку нечётному числу других чётно.
    
    \subsubsection{Пример}
    \begin{wrapfigure}{r}{6	cm}
        \vspace*{-1.3cm}
        \raggedright
       \begin{tikzpicture}[scale=1]
        \SetGraphUnit{1}
        \Vertex[Math] {v_1}
        \Vertex[Math, x=3,y=1]{v_2}
        \Vertex[Math, x=3,y=0]{v_3}
        \Vertex[Math, x=2,y=-1.5]{v_4}
        \Edge[style={-}, label=$2$](v_1)(v_2)
        \Edge[style={-}, label=$3$](v_1)(v_3)
        \Edge[style={-}, label=$4$](v_1)(v_4)
        \Loop[dist=1.5cm,dir=WE,style={-}, label=$1$, labelstyle={left=1pt}](v_1)
        \tikzset{EdgeStyle/.style = {-,bend right}}
        \Edge[label=$5$](v_1)(v_4)
        \end{tikzpicture}
    \vspace{-2cm}
    \end{wrapfigure}
    $\deg(v_1) = 6$, $\deg(v_2) = 1$, $\deg(v_3) = 1$, $\deg(v_4) =~2$;\\
    $\displaystyle\sum_{i=1}^{4} \deg(v_i) = 6 + 1 + 1 + 2 = 10 = 2 * 5 = 2 \cdot |E|$.
    
    \subsection{Следствие из леммы о рукопожатиях}
    В любом графе количество нечётных вершин (вершин нечётной степени) всегда чётно.
    \[ \sum_{i=1}^{n} \deg(v_i) = \smashoperator{\sum_{\deg(V_j)\scriptscriptstyle{\text{ - четн.}}}} \deg(v_j) + \smashoperator{\sum_{\deg(V_k)\scriptscriptstyle{\text{ - нечетн.}}}} \deg(v_k) = 2m.\]

    \subsection{Примеры графов}
    \subsubsection{Вполне несвязный граф}
    Вполне несвязный граф (пустой граф, нуль-граф) $G = (V, E)$, где $|V| = n$, $|E| = 0$ --- регулярный граф (опред. \ref{reggraph}) степени 0, то есть граф без рёбер.
    \begin{figure}[H]
        \centering
        \begin{tikzpicture}[scale=1]
        \SetGraphUnit{1}
        \Vertex[Math] {v_1}
        \Vertex[Math, x=2,y=0]{v_2}
        \Vertex[Math, x=2,y=2]{v_3}
        \Vertex[Math, x=0,y=2]{v_4}
        \end{tikzpicture}
    \end{figure}
    \clearpage
    
    \subsubsection{Полный граф}
    Граф $K_n = (V, E)$ с $|V| = n$ называется полным, если для каждой пары вершин $v_1$, $v_2$, существует ребро, инцидентное $v_1$ и инцидентное $v_2$ (каждая вершина соединена ребром с любой другой вершиной).
    \[ |E(K_n)| = C_n^2 = \frac{n(n-1)}{2},  \text{ по лемме \ref{handsshake}}\]
    \begin{figure}[H]
    \vspace*{-15pt}
    \centering
        \begin{tikzpicture}[scale=1.5]
            \SetGraphUnit{1}
            \begin{scope}[rotate=18]
                \Vertices[Lpos=45]{circle}{A, B, C, D, E}
            \end{scope}
            \Edges(A,B,A,C,A,D,A,E,B,C,B,D,B,E,D,C,D,E,C,E)
        \end{tikzpicture}
        \caption*{$K_5$}
    \end{figure}

    \subsubsection{Двудольный граф и полный двудольный граф}
    Двудольный граф (или биграф, или чётный граф) — это граф $G(V,E)$, такой, что множество вершин $V$ разбито на два непересекающихся подмножества $V_1$ и $V_2$, причём всякое ребро $E$ инцидентно вершине из $V_1$ и вершине из $V_2$ (то есть соединяет вершину из $V_1$ с вершиной из $V_2$). Множества $V_1$ и $V_2$ называются <<долями>> двудольного графа. 
    
    Полный двудольный граф --- специальный вид двудольного графа, у которого любая вершина первой доли соединена со всеми вершинами второй доли вершин. 
    \begin{figure}[H]
        \centering	
        \begin{tikzpicture}[scale=1.5]
        \SetGraphUnit{1}
        \Vertex[Math] {v_1}
        \Vertex[Math, x=1,y=-1]{v_2}
        \Vertex[Math, x=2,y=0]{v_3}
        \Vertex[Math, x=3,y=-1]{v_4}
        \Vertex[Math, x=4,y=0]{v_5}
        \Edges(v_4, v_1, v_2, v_3, v_4, v_5, v_2)
        \draw [decorate,decoration={brace,amplitude=10pt,raise=10pt},yshift=2pt]
        (0,0) -- (4,0) node [red,midway,xshift=0, yshift=30pt] {Множество $V_1$};
        
        \draw [decorate,decoration={brace,amplitude=10pt, mirror, raise=10pt},yshift=-2pt]
        (1,-1) -- (3,-1) node [red,midway,xshift=0,yshift=-30pt] {Множество $V_2$};
        \end{tikzpicture}
        \caption*{Полный двудольный граф $K_{3,2}$}
    \end{figure}

\clearpage
\section{Лекция 8}

\subsection{Маршрут}

Маршрут длинны $k$:
$$ v_0 e_1 v_1 e_2 v_2 \dots v_{k-1} e_k v_k $$

Замечания:
\begin{enumerate}
\item Если граф простой, то из маршрута можно отбросить рёбера $e$.
\item Маршрут длинны 0 --- вершина.
\end{enumerate}

\subsection{Замкнутый маршрут}

Маршрут называется замкнутым, если начальная и конечная вершины совпадают.

\subsection{Цепь}
Незамкнутый маршрут называется цепью, если рёбра попарно различны.

\subsection{Простая цепь}
Цепь, где вершины не повторяются называется простой.

\subsection{Цикл}
Замкнутый маршрут называется циклом, если нет повторяющихся рёбер.

\subsection{Простой цикл}

Цикл называется простым, если нет повторяющихся вершин.

\begin{center}
\begin{tikzpicture}[node distance = 2cm, auto]
    % Place nodes
    % \node [block] (init) {initialize model};
    \node [cloud] (marsh) {Маршрут};
    \node [below of=marsh] (empt) {};
    \node [block, left of=empt] (zam) {Замкнутый};
    \node [block, below of=zam] (loop) {Цикл};
    \node [block, below of=loop] (easy) {Простой цикл};
    \node [block2, right of=empt] (nezam) {Незамкнутый};
    \node [block2, below of=nezam] (cep) {Цепь};
    \node [block2, below of=cep] (easy2) {Простая цепь};
    \path [line] (marsh) -| (zam);
    \path [line] (marsh) -| (nezam);
    \path [line] (zam) -- (loop);
    \path [line] (loop) -- (easy);
    \path [line] (nezam) -- (cep);
    \path [line] (cep) -- (easy2);
\end{tikzpicture}
\end{center}
\subsection{Регулярный граф} \label{reggraph}
Граф $G = (V,E)$ -- регулярный, если степени вершин равны $$ \forall v \in V: \deg{v} = e$$

\subsection{Расстояние}

Расстояние $d(u,v)$ --- кратчайший маршрут от $u$ до $v$.

\subsection{Диаметр}

Диаметр графа --- расстояние между самыми удалёнными вершинами.

\subsection{Подграф}

Подграф $G' = (V',E')$ состоит из части вершин ($V' \subset V$) и рёбер ($E' 
\subset E$) графа $G = (V,E)$

\subsection{Дополнительный граф}

$G = G(V,E)$ --- простой граф, граф $\bar{G} = (V,\bar{E})$ --- дополнительный, если
$$ \{u,v\} \in E \Leftrightarrow \{u,v\} \notin \bar{E}$$

\definecolor{uququq}{rgb}{0.25098039215686274,0.25098039215686274,0.25098039215686274}
\begin{center}
\begin{tikzpicture}[line cap=round,line join=round,>=triangle 45,x=0.35cm,y=0.35cm]
\clip(0.,0.) rectangle (16.,12.);
\draw [line width=2.pt,,color=uququq] (0.7597143250020792,10.757406692951177)-- (0.7597143250020792,6.757406692951177);
\draw [line width=2.pt,,color=uququq] (4.759714325002079,10.757406692951177)-- (4.759714325002079,6.757406692951177);
\draw [line width=2.pt,,color=uququq] (0.7597143250020792,10.757406692951177)-- (4.759714325002079,6.757406692951177);
\draw [line width=2.pt,,color=uququq] (0.7597143250020792,6.757406692951177)-- (4.759714325002079,10.757406692951177);
\draw [line width=2.pt,,color=uququq] (0.7597143250020792,4.757406692951177)-- (4.759714325002079,4.757406692951177);
\draw [line width=2.pt,,color=uququq] (0.7597143250020792,0.7574066929511769)-- (4.759714325002079,0.7574066929511769);
\draw [line width=2.pt,,color=uququq] (8.75971432500208,10.757406692951177)-- (8.75971432500208,6.757406692951177);
\draw [line width=2.pt,,color=uququq] (12.75971432500208,10.757406692951177)-- (12.75971432500208,6.757406692951177);
\draw [line width=2.pt,,color=uququq] (8.75971432500208,10.757406692951177)-- (12.75971432500208,6.757406692951177);
\draw [line width=2.pt,,color=uququq] (8.75971432500208,6.757406692951177)-- (12.75971432500208,10.757406692951177);
\draw [line width=2.pt,,color=uququq] (8.75971432500208,6.757406692951177)-- (12.75971432500208,6.757406692951177);
\draw [line width=2.pt,,color=uququq] (8.75971432500208,10.757406692951177)-- (12.75971432500208,10.757406692951177);
\draw (5.389263418251354,9.85939059511299) node[anchor=north west] {$K_{2,2}$};
\draw (5.340554895000522,4.01436780501313) node[anchor=north west] {$\bar{K}_{2,2}$};
\draw (13.280044184886188,9.85939059511299) node[anchor=north west] {$N_{2,2}$};
\draw (13.231335661635356,3.8682422352606336) node[anchor=north west] {$\bar{N}_{2,2}$};
\begin{scriptsize}
\draw [fill=uququq] (0.7597143250020792,10.757406692951177) circle (4.5pt);
\draw [fill=uququq] (0.7597143250020792,6.757406692951177) circle (4.5pt);
\draw [fill=uququq] (4.759714325002079,10.757406692951177) circle (4.5pt);
\draw [fill=uququq] (4.759714325002079,6.757406692951177) circle (4.5pt);
\draw [fill=uququq] (0.7597143250020792,4.757406692951177) circle (4.5pt);
\draw [fill=uququq] (0.7597143250020792,0.7574066929511769) circle (4.5pt);
\draw [fill=uququq] (4.759714325002079,4.757406692951177) circle (4.5pt);
\draw [fill=uququq] (4.759714325002079,0.7574066929511769) circle (4.5pt);
\draw [fill=uququq] (8.75971432500208,10.757406692951177) circle (4.5pt);
\draw [fill=uququq] (8.75971432500208,6.757406692951177) circle (4.5pt);
\draw [fill=uququq] (12.75971432500208,10.757406692951177) circle (4.5pt);
\draw [fill=uququq] (12.75971432500208,6.757406692951177) circle (4.5pt);
\draw [fill=uququq] (8.75971432500208,4.757406692951177) circle (4.5pt);
\draw [fill=uququq] (8.75971432500208,0.7574066929511769) circle (4.5pt);
\draw [fill=uququq] (12.75971432500208,4.757406692951177) circle (4.5pt);
\draw [fill=uququq] (12.75971432500208,0.7574066929511769) circle (4.5pt);
\end{scriptsize}
\end{tikzpicture}
\end{center}

\subsection{Связный граф}
Граф $G=(V,E)$ называется связным, если любые две вершины соединеный маршрутом.

\subsection{Компонент связности}
$G=(V,E)$ -- компонент связности, если он является макимальным по включению связным подграфом.

\subsection{Мост}
$e \in E$ -- мост (перешеек), если после его удаления количество компонентов связности в исходном графе увеличивается.

\subsection{Разделяющая точка}
$v \in V$ -- разделяющая точка, если удаление этой точки приводит к увеличению компонентов связности в исходном графе.

\subsection{Изоморфизм графа}
$G_1=(V_1,E_1),\ G_2=(V_2,E_2)$ -- простые.

$G_1 \cong G_2$ -- изоморфны, если существует взаимосвязь: $\exists \varphi:V_1 \rightarrow V_2$, такая что: $$\{u,v\}\in E_1 \Leftrightarrow \{\varphi(u),\varphi(v)\}\in E_2$$
\begin{center}
\begin{tikzpicture}[line cap=round,line join=round,>=triangle 45,x=0.35cm,y=0.35cm]
\clip(0.,0.) rectangle (27.5,22.);
\draw [line width=2.pt,,color=uququq] (6.246897664223317,16.35945821084448)-- (1.246897664223317,11.359458210844474);
\draw [line width=2.pt,,color=uququq] (6.246897664223317,16.35945821084448)-- (11.246897664223317,11.359458210844474);
\draw [line width=2.pt,,color=uququq] (1.246897664223317,1.3594582108444744)-- (6.246897664223317,6.359458210844474);
\draw [line width=2.pt,,color=uququq] (1.246897664223317,11.359458210844474)-- (11.246897664223317,11.359458210844474);
\draw [line width=2.pt,,color=uququq] (6.246897664223317,6.359458210844474)-- (11.246897664223317,11.359458210844474);
\draw (6.471646426115806,21.877100178321726) node[anchor=north west] {$G_1$};
\draw [line width=2.pt,,color=uququq] (16.24689766422332,11.359458210844474)-- (16.24689766422332,16.35945821084448);
\draw [line width=2.pt,,color=uququq] (16.24689766422332,11.359458210844474)-- (21.24689766422332,11.359458210844474);
\draw [line width=2.pt,,color=uququq] (26.24689766422332,1.3594582108444744)-- (21.24689766422332,6.359458210844474);
\draw [line width=2.pt,,color=uququq] (16.24689766422332,16.35945821084448)-- (21.24689766422332,11.359458210844474);
\draw [line width=2.pt,,color=uququq] (21.24689766422332,6.359458210844474)-- (21.24689766422332,11.359458210844474);
\draw (16.445272429758074,21.877100178321726) node[anchor=north west] {$G_2$};
\begin{scriptsize}
\draw [fill=uququq] (6.246897664223317,16.35945821084448) circle (4.5pt);
\draw[color=uququq] (7.1365548263586245,18.129434649680405) node {$u_2$};
\draw [fill=uququq] (1.246897664223317,11.359458210844474) circle (4.5pt);
\draw[color=uququq] (2.179964933639437,13.11239853875735) node {$u_1$};
\draw [fill=uququq] (11.246897664223317,11.359458210844474) circle (4.5pt);
\draw[color=uququq] (12.153590937281704,13.11239853875735) node {$u_3$};
\draw [fill=uququq] (1.246897664223317,1.3594582108444744) circle (4.5pt);
\draw[color=uququq] (2.179964933639437,3.13877253511513) node {$u_5$};
\draw [fill=uququq] (6.246897664223317,6.359458210844474) circle (4.5pt);
\draw[color=uququq] (7.1365548263586245,8.095362427834294) node {$u_4$};
\draw [fill=uququq] (16.24689766422332,11.359458210844474) circle (4.5pt);
\draw[color=uququq] (17.170627048204786,13.11239853875735) node {$v_2$};
\draw [fill=uququq] (16.24689766422332,16.35945821084448) circle (4.5pt);
\draw[color=uququq] (17.170627048204786,18.129434649680405) node {$v_1$};
\draw [fill=uququq] (21.24689766422332,11.359458210844474) circle (4.5pt);
\draw[color=uququq] (22.127216940923972,13.11239853875735) node {$v_3$};
\draw [fill=uququq] (26.24689766422332,1.3594582108444744) circle (4.5pt);
\draw[color=uququq] (27.144253051847052,3.13877253511513) node {$v_5$};
\draw [fill=uququq] (21.24689766422332,6.359458210844474) circle (4.5pt);
\draw[color=uququq] (22.127216940923972,8.095362427834294) node {$v_4$};
\end{scriptsize}
\end{tikzpicture}
\end{center}

\subsection{Необходимые признаки изоморфности:}
$G_1 \cong G_2$:
\begin{enumerate}
\item $|V_1| = |V_2|$
\item $|E_1| = |E_2|$
\item Набор степеней вершин одинаков.
\end{enumerate}
\end{document}
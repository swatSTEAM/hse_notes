
\documentclass[a4paper,12pt]{article} % добавить leqno в [] для нумерации слева
\usepackage[top=2 cm, bottom=2cm, left=3cm, right=1.5cm]{geometry}
%%% Работа с русским языком
\usepackage{cmap}					% поиск в PDF
\usepackage{mathtext} 				% русские буквы в формулах
\usepackage[T2A]{fontenc}			% кодировка
\usepackage[utf8]{inputenc}			% кодировка исходного текста
\usepackage[english,russian]{babel}	% локализация и переносы

%%% Дополнительная работа с математикой
\usepackage{amsmath,amsfonts,amssymb,amsthm,mathtools} % AMS
\usepackage{icomma} % "Умная" запятая: $0,2$ --- число, $0, 2$ --- перечисление

%% Номера формул
%\mathtoolsset{showonlyrefs=true} % Показывать номера только у тех формул, на которые есть \eqref{} в тексте.

%% Шрифты
\usepackage{euscript}	 % Шрифт Евклид
\usepackage{mathrsfs} % Красивый матшрифт

%% Свои команды\textbf{}
\DeclareMathOperator{\sgn}{\mathop{sgn}}

%% Перенос знаков в формулах (по Львовскому)
\newcommand*{\hm}[1]{#1\nobreak\discretionary{}
{\hbox{$\mathsurround=0pt #1$}}{}}
\usepackage{indentfirst}
% \setlength{\parindent}{1.25cm}
\usepackage{tikz}
\usepackage{mathrsfs}
\usetikzlibrary{arrows}
\usetikzlibrary{shapes,arrows}
% \usepackage{pgfplots}
% \pgfplotsset{compat=1.9}
\tikzstyle{decision} = [diamond, draw, fill=blue!20, 
    text width=4.5em, text badly centered, node distance=3cm, inner sep=0pt]
\tikzstyle{block} = [rectangle, draw, fill=blue!20, 
    text width=7em, text centered, rounded corners, minimum height=2em]
\tikzstyle{block2} = [rectangle, draw, fill=yellow!20, 
    text width=7em, text centered, rounded corners, minimum height=2em]
\tikzstyle{line} = [draw, -latex']
\tikzstyle{cloud} = [draw, ellipse,fill=red!20, node distance=4cm,
    minimum height=2em]
    
\begin{document}
\setcounter{section}{7}
\section{Лекция 8}

\subsection{Граф}

$G(V,E)$ --- граф, где $V$ --- множество вершин, $E$ --- множество ребер.

\subsection{Маршрут}

Маршрут длинны $k$:
$$ v_0 e_1 v_1 e_2 v_2 \dots v_{k-1} e_k v_k $$

Замечания:
\begin{enumerate}
\item Если граф простой, то из маршрута можно отбросить рёбера $e$.
\item Маршрут длинны 0 --- вершина.
\end{enumerate}

\subsection{Замкнутый маршрут}

Маршрут называется замкнутым, если начальная и конечная вершины совпадают.

\subsection{Цепь}
Незамкнутый маршрут называется цепью, если рёбра попарно различны.

\subsection{Простая цепь}
Цепь, где вершины не повторяются называется простой.

\subsection{Цикл}
Замкнутый маршрут называется циклом, если нет повторяющихся рёбер.

\subsection{Простой цикл}

Цикл называется простым, если нет повторяющихся вершин.

\begin{center}
\begin{tikzpicture}[node distance = 2cm, auto]
    % Place nodes
    % \node [block] (init) {initialize model};
    \node [cloud] (marsh) {Маршрут};
    \node [below of=marsh] (empt) {};
    \node [block, left of=empt] (zam) {Замкнутый};
    \node [block, below of=zam] (loop) {Цикл};
    \node [block, below of=loop] (easy) {Простой цикл};
    \node [block2, right of=empt] (nezam) {Незамкнутый};
    \node [block2, below of=nezam] (cep) {Цепь};
    \node [block2, below of=cep] (easy2) {Простая цепь};
    \path [line] (marsh) -| (zam);
    \path [line] (marsh) -| (nezam);
    \path [line] (zam) -- (loop);
    \path [line] (loop) -- (easy);
    \path [line] (nezam) -- (cep);
    \path [line] (cep) -- (easy2);
\end{tikzpicture}
\end{center}
\subsection{Регулярный граф}
Граф $G = (V,E)$ -- регулярный, если степени вершин равны $$ \forall v \in V: \deg{v} = e$$

\subsection{Расстояние}

Расстояние $d(u,v)$ --- кратчайший маршрут от $u$ до $v$.

\subsection{Диаметр}

Диаметр графа --- расстояние между самыми удалёнными вершинами.

\subsection{Подграф}

Подграф $G' = (V',E')$ состоит из части вершин ($V' \subset V$) и рёбер ($E' 
\subset E$) графа $G = (V,E)$

\subsection{Дополнительный граф}

$G = G(V,E)$ --- простой граф, граф $\bar{G} = (V,\bar{E})$ --- дополнительный, если
$$ \{u,v\} \in E \Leftrightarrow \{u,v\} \in \bar{E}$$

\definecolor{uququq}{rgb}{0.25098039215686274,0.25098039215686274,0.25098039215686274}
\begin{center}
\begin{tikzpicture}[line cap=round,line join=round,>=triangle 45,x=0.35cm,y=0.35cm]
\clip(0.,0.) rectangle (16.,12.);
\draw [line width=2.pt,,color=uququq] (0.7597143250020792,10.757406692951177)-- (0.7597143250020792,6.757406692951177);
\draw [line width=2.pt,,color=uququq] (4.759714325002079,10.757406692951177)-- (4.759714325002079,6.757406692951177);
\draw [line width=2.pt,,color=uququq] (0.7597143250020792,10.757406692951177)-- (4.759714325002079,6.757406692951177);
\draw [line width=2.pt,,color=uququq] (0.7597143250020792,6.757406692951177)-- (4.759714325002079,10.757406692951177);
\draw [line width=2.pt,,color=uququq] (0.7597143250020792,4.757406692951177)-- (4.759714325002079,4.757406692951177);
\draw [line width=2.pt,,color=uququq] (0.7597143250020792,0.7574066929511769)-- (4.759714325002079,0.7574066929511769);
\draw [line width=2.pt,,color=uququq] (8.75971432500208,10.757406692951177)-- (8.75971432500208,6.757406692951177);
\draw [line width=2.pt,,color=uququq] (12.75971432500208,10.757406692951177)-- (12.75971432500208,6.757406692951177);
\draw [line width=2.pt,,color=uququq] (8.75971432500208,10.757406692951177)-- (12.75971432500208,6.757406692951177);
\draw [line width=2.pt,,color=uququq] (8.75971432500208,6.757406692951177)-- (12.75971432500208,10.757406692951177);
\draw [line width=2.pt,,color=uququq] (8.75971432500208,6.757406692951177)-- (12.75971432500208,6.757406692951177);
\draw [line width=2.pt,,color=uququq] (8.75971432500208,10.757406692951177)-- (12.75971432500208,10.757406692951177);
\draw (5.389263418251354,9.85939059511299) node[anchor=north west] {$K_{2,2}$};
\draw (5.340554895000522,4.01436780501313) node[anchor=north west] {$\bar{K}_{2,2}$};
\draw (13.280044184886188,9.85939059511299) node[anchor=north west] {$N_{2,2}$};
\draw (13.231335661635356,3.8682422352606336) node[anchor=north west] {$\bar{N}_{2,2}$};
\begin{scriptsize}
\draw [fill=uququq] (0.7597143250020792,10.757406692951177) circle (4.5pt);
\draw [fill=uququq] (0.7597143250020792,6.757406692951177) circle (4.5pt);
\draw [fill=uququq] (4.759714325002079,10.757406692951177) circle (4.5pt);
\draw [fill=uququq] (4.759714325002079,6.757406692951177) circle (4.5pt);
\draw [fill=uququq] (0.7597143250020792,4.757406692951177) circle (4.5pt);
\draw [fill=uququq] (0.7597143250020792,0.7574066929511769) circle (4.5pt);
\draw [fill=uququq] (4.759714325002079,4.757406692951177) circle (4.5pt);
\draw [fill=uququq] (4.759714325002079,0.7574066929511769) circle (4.5pt);
\draw [fill=uququq] (8.75971432500208,10.757406692951177) circle (4.5pt);
\draw [fill=uququq] (8.75971432500208,6.757406692951177) circle (4.5pt);
\draw [fill=uququq] (12.75971432500208,10.757406692951177) circle (4.5pt);
\draw [fill=uququq] (12.75971432500208,6.757406692951177) circle (4.5pt);
\draw [fill=uququq] (8.75971432500208,4.757406692951177) circle (4.5pt);
\draw [fill=uququq] (8.75971432500208,0.7574066929511769) circle (4.5pt);
\draw [fill=uququq] (12.75971432500208,4.757406692951177) circle (4.5pt);
\draw [fill=uququq] (12.75971432500208,0.7574066929511769) circle (4.5pt);
\end{scriptsize}
\end{tikzpicture}
\end{center}

\subsection{Связный граф}
Граф $G=(V,E)$ называется связным, если любые две вершины соединеный маршрутом.

\subsection{Компонент связности}
$G=(V,E)$ -- компонент связности, если он является макимальным по включению связным подграфом.

\subsection{Мост}
$e \in E$ -- мост (перешеек), если после его удаления количество компонентов связности в исходном графе увеличивается.

\subsection{Разделяющая точка}
$v \subset V$ -- разделяющая точка, если удаление этой точки приводит к увеличению компонентов связности в исходном графе.

\subsection{Изоморфизм графа}
$G_1=(V_1,E_1),\ G_2=(V_2,E_2)$ -- простые.

$G_1 \cong G_2$ -- изоморфны, если существует взаимосвязь: $\exists \varphi:V_1 \rightarrow V_2$, такая что: $$\{u,v\}\in E_1 \Leftrightarrow \{\varphi(u),\varphi(v)\}\in E_2$$
\begin{center}
\begin{tikzpicture}[line cap=round,line join=round,>=triangle 45,x=0.35cm,y=0.35cm]
\clip(0.,0.) rectangle (27.5,22.);
\draw [line width=2.pt,,color=uququq] (6.246897664223317,16.35945821084448)-- (1.246897664223317,11.359458210844474);
\draw [line width=2.pt,,color=uququq] (6.246897664223317,16.35945821084448)-- (11.246897664223317,11.359458210844474);
\draw [line width=2.pt,,color=uququq] (1.246897664223317,1.3594582108444744)-- (6.246897664223317,6.359458210844474);
\draw [line width=2.pt,,color=uququq] (1.246897664223317,11.359458210844474)-- (11.246897664223317,11.359458210844474);
\draw [line width=2.pt,,color=uququq] (6.246897664223317,6.359458210844474)-- (11.246897664223317,11.359458210844474);
\draw (6.471646426115806,21.877100178321726) node[anchor=north west] {$G_1$};
\draw [line width=2.pt,,color=uququq] (16.24689766422332,11.359458210844474)-- (16.24689766422332,16.35945821084448);
\draw [line width=2.pt,,color=uququq] (16.24689766422332,11.359458210844474)-- (21.24689766422332,11.359458210844474);
\draw [line width=2.pt,,color=uququq] (26.24689766422332,1.3594582108444744)-- (21.24689766422332,6.359458210844474);
\draw [line width=2.pt,,color=uququq] (16.24689766422332,16.35945821084448)-- (21.24689766422332,11.359458210844474);
\draw [line width=2.pt,,color=uququq] (21.24689766422332,6.359458210844474)-- (21.24689766422332,11.359458210844474);
\draw (16.445272429758074,21.877100178321726) node[anchor=north west] {$G_2$};
\begin{scriptsize}
\draw [fill=uququq] (6.246897664223317,16.35945821084448) circle (4.5pt);
\draw[color=uququq] (7.1365548263586245,18.129434649680405) node {$u_2$};
\draw [fill=uququq] (1.246897664223317,11.359458210844474) circle (4.5pt);
\draw[color=uququq] (2.179964933639437,13.11239853875735) node {$u_1$};
\draw [fill=uququq] (11.246897664223317,11.359458210844474) circle (4.5pt);
\draw[color=uququq] (12.153590937281704,13.11239853875735) node {$u_3$};
\draw [fill=uququq] (1.246897664223317,1.3594582108444744) circle (4.5pt);
\draw[color=uququq] (2.179964933639437,3.13877253511513) node {$u_5$};
\draw [fill=uququq] (6.246897664223317,6.359458210844474) circle (4.5pt);
\draw[color=uququq] (7.1365548263586245,8.095362427834294) node {$u_4$};
\draw [fill=uququq] (16.24689766422332,11.359458210844474) circle (4.5pt);
\draw[color=uququq] (17.170627048204786,13.11239853875735) node {$v_2$};
\draw [fill=uququq] (16.24689766422332,16.35945821084448) circle (4.5pt);
\draw[color=uququq] (17.170627048204786,18.129434649680405) node {$v_1$};
\draw [fill=uququq] (21.24689766422332,11.359458210844474) circle (4.5pt);
\draw[color=uququq] (22.127216940923972,13.11239853875735) node {$v_3$};
\draw [fill=uququq] (26.24689766422332,1.3594582108444744) circle (4.5pt);
\draw[color=uququq] (27.144253051847052,3.13877253511513) node {$v_5$};
\draw [fill=uququq] (21.24689766422332,6.359458210844474) circle (4.5pt);
\draw[color=uququq] (22.127216940923972,8.095362427834294) node {$v_4$};
\end{scriptsize}
\end{tikzpicture}
\end{center}

\subsection{Необходимые признаки изоморфности:}
$G_1 \cong G_2$:
\begin{enumerate}
\item $|V_1| = |V_2|$
\item $|E_1| = |E_2|$
\item Набор степеней вершин одинаков.
\end{enumerate}
\end{document}
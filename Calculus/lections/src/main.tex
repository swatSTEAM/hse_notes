\documentclass[a4paper, 12pt]{article}

\usepackage{cmap}					% поиск в PDF
\usepackage{mathtext} 				% русские буквы в формулах
\usepackage[T2A]{fontenc}			% кодировка
\usepackage[utf8]{inputenc}			% кодировка исходного текста
\usepackage[english,russian]{babel}	% локализация и переносы
\usepackage{hyphenat}
\hyphenation{мате-мати-ка восста-навливать}
\usepackage[top=2cm, bottom=2cm, left=2cm, right=2cm]{geometry}
\usepackage[bottom]{footmisc} % сноски всегда внизу
\usepackage{indentfirst} % отступ параграфа
\parindent=1cm

%%% Дополнительная работа с математикой
\usepackage{amsmath,amsfonts,amssymb,amsthm,mathtools}
\usepackage{soul}   % Поддержка переносоустойчивых подчёркиваний и зачёркиваний
\usepackage{icomma} % Запятая в десятичных дробях

%% Номера формул
%\mathtoolsset{showonlyrefs=true} % Показывать номера только у тех формул, на которые есть \eqref{} в тексте.

%% Перенос знаков в формулах (по Львовскому)
\newcommand*{\hm}[1]{#1\nobreak\discretionary{}
    {\hbox{$\mathsurround=0pt #1$}}{}}

\usepackage{array,tabularx,tabulary,booktabs} % Дополнительная работа с таблицами
\usepackage{longtable} % Длинные таблицы
\usepackage{multirow,makecell,array} % Улучшенное форматирование таблиц

% Содержание с точками
\usepackage{etoolbox}
\makeatletter
\patchcmd{\l@section}
{\hfil}
{\leaders\hbox{\normalfont$\m@th\mkern \@dotsep mu\hbox{.}\mkern \@dotsep mu$}\hfill}
{}{}

\bibliographystyle{utf8gost71u} % Оформляем библиографию по ГОСТ 7.1 (ГОСТ Р 7.0.11-2011, 5.6.7)
\renewcommand{\@biblabel}[1]{#1.} % Заменяем библиографию с квадратных скобок на точку
\makeatother

\usepackage{graphicx} % Для вставки рисунков
\graphicspath{{images/}} % Пути к изображениям
\setlength\fboxsep{3pt} % Отступ рамки \fbox{} от рисунка
\setlength\fboxrule{1pt} % Толщина линий рамки \fbox{}
\usepackage{wrapfig} % Обтекание рисунков текстом


\usepackage{hyperref}
\usepackage[usenames,dvipsnames,svgnames,table,rgb]{xcolor}
\hypersetup{ % Гиперссылки
    unicode=true, % русские буквы в раздела PDF
    pdftitle={Заголовок}, % Заголовок
    pdfauthor={Автор}, % Автор
    pdfsubject={Тема}, % Тема
    pdfcreator={Создатель}, % Создатель
    pdfproducer={Производитель}, % Производитель
    pdfkeywords={keyword1} {key2} {key3}, % Ключевые слова
    colorlinks=true, % false: ссылки в рамках; true: цветные ссылки
    linkcolor=black, % внутренние ссылки
    citecolor=black, % на библиографию
    filecolor=magenta, % на файлы
    urlcolor=cyan % на URL
}

\usepackage{tikz}

\begin{document}
    \tableofcontents
    \clearpage
    \section{Лекция 1}
    \subsection{Примеры рядов}
    Числовой ряд:
    \begin{equation}
        \sum_{n=1}^{\infty} a_n = \lim_{n \to \infty} S_n, \text{ где } S_n = a_1 + a_2 + \cdots + a_n.
    \end{equation}

    Ряд Тейлора:
    \begin{equation}
        f(x_0) = \sum_{n=0}^{\infty} \frac{f^{(n)}(x_0)}{n!}(x-x_0)^n, \text{ при } x_0 \in \mathbb{R} \text{ и } f \in c^{\infty}(O_h(x_0)).
    \end{equation}

    \subsection{Ряды Фурье}
    Другое представление сложных функций через простые - ряды Фурье.

    Если f(x) - $2\pi$-периодическая функция на $\mathbb{R}$, тогда $f(x) = a_0 + \sum_{n=1}^{\infty} a_ncos(nx) +~b_nsin(nx),$ где $a_0$, $a_n$, $b_n$ - коэффициенты Фурье, зависящие от функции $f(x)$.

    Этот ряд применяется к изучению периодических (колебательных) повторяющихся процессов в природе. Например, сигналы (звуковые, радио, теле), переменный электроток, а также в теории дифференциальных уравнений.

    \subsection{Гармонические колебания}
    Уравнение гармонических колебаний:
    \begin{equation}\label{eq:garmonickoleb}
        f(t) = Asin(\omega t + \varphi),
    \end{equation}
    где $t$ - время, $A > 0$ - амплитуда колебаний, $\omega$ - круговая частота, $T = \frac{2\pi}{\omega}$ - период, $\varphi$ - начальная фаза.

    \setlength\intextsep{-9pt}
    \begin{wrapfigure}[6]{r}[20pt]{0cm}
        \begin{tikzpicture}
        \draw (0,0) circle [radius=1.5];
        \draw (0.3,0.3) to [out=0,in=90] (0.5, 0);
        \node at (0.65, 0.25) {$\varphi$};
        \draw[->] (-2,0) -- (2,0) node [right] {$x$};
        \draw[->] (0,-2) -- (0,2) node [above] {$y$};
        \draw[] (0,0) -- (1.06,1.06);
        \draw[dotted] (1.06,1.06) -- (0,1.06) node [red, xshift=-0.15cm] {$a$};
        \draw[red, very thick] (1.06,1.06) -- (1.06,0);
        \draw[blue, very thick] (0,0) -- (1.06,0) node [blue, yshift=-0.2cm] {$b$};
        \end{tikzpicture}
    \end{wrapfigure}

    Так-как $sin\varphi = \frac{a}{\sqrt{a^2 + b^2}}$ и $cos\varphi = \frac{b}{\sqrt{a^2 + b^2}}$, тогда уравнение (\ref{eq:garmonickoleb}) можно переписать так:
    \begin{equation*}
    \begin{aligned}
        f(t) &= Asin(\omega t + \varphi) = Asin(\varphi)cos(\omega t) + Acos(\varphi)sin(\omega t) =\\
        &= \sqrt{a^2 + b^2}\left(\frac{a}{\sqrt{a^2 + b^2}}cos(\omega t) + \frac{b}{\sqrt{a^2 + b^2}}sin(\omega t)\right), a \neq 0, b \neq 0.
    \end{aligned}
    \end{equation*}

    \subsection{Что можно делать с рядами Фурье?}
    \begin{enumerate}
        \item Выделить из исходного сигнала его составляющую заданной частоте.
        \item Выделение сигнала соответствующего диапазона частот.
    \end{enumerate}

    Замечено, что конечные суммы $\sum_{n=1}^{N} a_n cos(n\omega x) + b_nsin(n\omega x)$ с частотами кратными $\omega$, обладают разнообразными свойствами. Вопрос: можно ли произвольные колебания, с основной частотой $\omega$, представить суммой гармонических колебаний с тем же периодом (не минимальным)? - Можно, если сумма бесконечна и добавлена постоянная составляющая.

    \subsection{Абстрактный аппарат для обоснования ряда Фурье}
    \subsubsection{Линейное векторное пространство}
    Линейным векторным пространством $X$ называется множество элементов $x$, на котором определены операции $+$ и $\times$ на числах: $\forall x,y \in X \Rightarrow \alpha x + \beta y: a, b \in \mathbb{R}$.
    \subsubsection{Аксиомы векторного пространства}
    \begin{enumerate}
        \item $\alpha(x + y) = \alpha x + \alpha y$
        \item $\alpha(\beta x) = (\alpha \beta) x$
        \item $(\alpha + \beta) x = \alpha x + \beta x x$
        \item $\exists$ $0 \in X : 0 + x = x$
    \end{enumerate}

    \subsubsection{Линейная зависимость и независимость векторов}
    Пусть $X \in \mathbb{R} : X = (x_1, x_2, ..., x_n) \text{, где }x_i \in \mathbb{R}.$ Тогда $x_1, x_2, ... , x_n \in X$ линейно независимы, если:
    \[\alpha_1 x_1 + \alpha_2 x_2 + ...  + \alpha_n x_n = 0 \Rightarrow \alpha_1 = \alpha_2 =  ...  = \alpha_n = 0.\]
    Если существует такая линейная комбинация с минимум одним $a_i \neq 0$, тогда $x_1, x_2, ... , x_n \in X$ линейно зависимы.

    \subsubsection{Евклидово векторное пространство}
    Линейное векторное пространство называется Евклидовым, если в нём задана операция скалярного произведения: $x, y \to (x, y)$ и удовлетворяет аксиомам:
    \begin{enumerate}
        \item $(x, y) = (y, x)$
        \item $(\alpha x + \beta y, z) = \alpha(x, z) + \beta (y, z)$
        \item $(x, x) \geqslant 0, \text{ причем, если } (x, x) = 0 \Leftrightarrow x = 0$
        \item Если $x, y \in \mathbb{R}^n : (x, y) = x_1 y_1 + x_2 y_2 + ... + x_n y_n$
    \end{enumerate}

    \subsubsection{Норма вектора}
    \[\Vert X \Vert = \sqrt{(x, x)} \Rightarrow \Vert X \Vert \geqslant 0.\]

    Свойства нормы в $X$:
    \begin{enumerate}
        \item $\Vert X \Vert = 0 \Leftrightarrow x = 0.$\\
        Доказательство:\\
        Так-как $\Vert x \Vert = \sqrt{(x, x)}$, то неравенство следует из скалярного произведения.
        \item $\Vert \alpha x \Vert = |\alpha| \cdot \Vert X \Vert$ \\
        Доказательство:\\
        $\Vert \alpha x \Vert = \sqrt{(\alpha x, \alpha x)} = \sqrt{\alpha^2 (x, x)} = |\alpha| \cdot \Vert x \Vert$.
        \item $|(x, y)| \leqslant \Vert x \Vert \cdot \Vert y \Vert$ (неравенство Коши-Буняковского)\\
        Доказательство:\\
        Возьмем $t \in \mathbb{R}$ и рассмотрим уравнение:
        \begin{align*}
            (x-ty, &x - ty) = (x, x) - t(x,y) - t(y, x) + t^2(y, y) = \\
            &= (x, x) - 2t(x,y) + t^2(y,y) = 0,
        \end{align*}
        если $y \neq 0$, то $(y, y) > 0$, тогда $\frac{D}{4} \leqslant 0$:
        \begin{align*}
            \frac{D}{4} = (x, y)^2 &- (x, x)(y, y) \leqslant 0;\\
            (x, y)^2 &\leqslant (x, x)(y, y);\\
            |(x, y)| &\leqslant \Vert x \Vert \cdot \Vert y \Vert.
        \end{align*}

        \item $\Vert x + y \Vert \leqslant  \Vert x \Vert + \Vert y \Vert$ (неравенство треугольника).\\
        Доказательство:
        \begin{align*}
            \Vert x + &y \Vert^2 = (x + y, x + y) = (x, x) + 2(x, y) + (y, y) \leqslant \\
            &\leqslant \Vert x \Vert^2 + \Vert y \Vert^2 + 2\Vert x \Vert \cdot \Vert y \Vert = (\Vert x \Vert + \Vert y \Vert)^2.
        \end{align*}
    \end{enumerate}

    Смысл нормы $\Vert x \Vert$ - длина вектора, $\Vert x - y \Vert$ - расстояние.

    Последовательность $X_n \in X$ (в евклидовом пространстве) называется сходящейся \\$\lim_{x \to \infty} X_n = x,$ где $x \in X$, в том смысле, что $\lim_{n \to \infty} \Vert x_n - x \Vert = 0$.

    \subsubsection{Плотное множество}
    Множество $E \subset X$ (в евклидовом пространстве) плотно, если $\forall x \in X$ $\exists$ $x_1, x_2, ..., x_n \in E : x_n \xrightarrow{n \to \infty} x$.

    \subsubsection{Ортогональная система векторов}
    Система векторов $x_1, x_2 ..., x_n$ (конечная или бесконечная) называется ортогональной, если $(x_i, x_j) = 0$, при $i \neq j$.

    \subsubsection{Ортонормированная система векторов}
    Система векторов $x_1, x_2 ..., x_n$ (конечная или бесконечная) называется ортонормированной, если она ортогональна и $\Vert x_j \Vert = 1$.

    Обозначим $C_2 [-\pi; \pi]$ множество функций, определенных на отрезке $[-\pi; \pi]$, таких что функция $f$ непрерывна на этом отрезке, кроме, быть может, конечного числа точек (разрыв типа <<скачок>>). Значения в точках разрыва и в точках $-\pi$, $\pi$ на функцию не влияют, т.е. функция и $C_2 [-\pi; \pi]$ различны в конечном числе точек - отождествляются.
\end{document}
